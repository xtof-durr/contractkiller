\problemname{Tueur à gage}

\begin{center}
\includegraphics[height=8cm]{contract-killer.png}
\end{center}


Dans le film "I Hired a Contract Killer", Henri Boulange a décidé de mettre fin à ses jours, et emploie un tueur à gages pour cela. Alors qu'il attend patiemment l'arrivé du tueur dans un café, il rencontre une vendeuse de fleurs. Immédiatement il en tombe amoureux et désormais veut rester en vie. N'ayant plus la possibilité d'entrer en contact avec le tueur embauché, il doit fuir.

S'en suit une course-poursuite qui peut être décrite ainsi. La ville est modélisée par un graphe, dont les sommets sont numérotés de 0 à $n-1$. D'abord Henri choisit un sommet, puis le tueur choisit un autre sommet. Le reste de la course-poursuite consiste en plusieurs rondes. Au début d'une ronde Henri peut choisir de rester sur son sommet ou de rejoindre un sommet voisin. Puis c'est au tour du tueur de choisir de rester sur son sommet ou de rejoindre un sommet voisin.  Si à ce moment là le tueur est sur le même sommet qu'Henri, alors il va le tuer. Sinon débute une nouvelle ronde. Aidez Henri à identifier les sommets sûrs qui lui garantissent de rester en vie s'il débute dans ces sommets. Plus précisément indiquez la somme des identifiants de ces sommets.

\section*{Entrée}
La première ligne de l'entrée consiste en deux entiers  $n$ et $m$, où $n$ indique le nombre de sommets et $m$ le nombre d'arêtes avec $2 \leq n \leq 1.000$ and $0 \leq m \leq 30.000$. S'en suivent $m$ lignes qui décrivent les arêtes. Chacune de ces lignes contient deux entiers $u$ et $v$ avec $0 \leq u < v < n$ indiquant une arête entre les sommets  $u$ et $v$.

\section*{Sortie}
Affichez un seul entier, la somme des indices des sommets sûrs.


\section*{Exemples}

Dans le premier exemple, les sommets $0,1,2,3,4$ sont sûrs, alors que les autres ne le sont pas. Par exemple si Henri choisi un sommet sur le cycle $5-6-8-5$, alors le tueur peut choisir un autre sommet sur ce cycle, et Henri ne peut pas éviter d'être à distance au plus 1 du tueur. En conséquence il sera tué.
\begin{center}
\includegraphics[width=8cm]{sample1.pdf}
\end{center}

Dans le deuxième exemple, seuls les sommets $7,8,9,10,11$ sont sûrs.
\begin{center}
\includegraphics[width=8cm]{sample2.pdf}
\end{center}


